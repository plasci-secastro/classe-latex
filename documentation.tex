\documentclass{plasci-secastro}
\usepackage{polyglossia,fontspec,xunicode,metalogo}
\usepackage{hyperref}
\usepackage{minted}
\usepackage{lipsum}
\setmainfont{Calibri}
\setmainlanguage{french}
\date{Campagne astro 2013}
\author{Maïeul Rouquette}
\title{\LaTeX\ pour compte rendu du secteur astronomie}
\begin{document}
\maketitle

Utilisation :
\begin{enumerate}
	\item Télécharger sur \url{https://github.com/plasci-secastro/classe-latex/archive/master.zip} la classe et les fichiers image liés.
	\item Les mettre dans un répertoire que \LaTeX\ peut trouver, par exemple le dossier avec le \verb+.tex.+ ou votre dossier de config local de \LaTeX\footnote{Voir \url{http://latex.developpez.com/faq/?page=LATEX_ADVANCED}.}.
	\item La classe se charge normalement avec \verb+\documenclass{plasci-secastro}+. Elle propose les fonctions suivantes :
		\begin{itemize}
			\item Entête et pied de page avec la barrette Planète-Sciences.
			\item Commande \verb+maketitle+ affichant les logos de Planète-Sciences et du Secteur Astro.
			\item Pied de page contenant le titre
		\end{itemize}
	\item Il faut en revanche charger ses packages perso pour la gestion des caractères, des polices et de la typo :
		\begin{itemize}
			\item Avec \XeLaTeX:
				\begin{minted}{tex}
\usepackage{polyglossia,fontspec,xunicode}
\setmainlanguage{french}
\setmainfont{Calibri}% Ou tout autre
				\end{minted}
			\item Avec \LaTeX, si on encode en Unicode \textbf{Ce qui est toujours mieux}:
				\begin{minted}{tex}
\usepackage[utf8]{inputenc}  
\usepackage[T1]{fontenc}
\usepackage[french]{babel}  
				\end{minted}
		\end{itemize}
	\item Évidemment vous pouvez aussi choisir d'ajouter d'autres packages.
	\item Pour améliorer la classe, demander un accès au compte GitHub à Maïeul.
\end{enumerate}
\end{document}
